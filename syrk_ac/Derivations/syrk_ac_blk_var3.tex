\documentclass[12pt]{article}

\usepackage{amssymb}
\usepackage{ifthen}
\usepackage[table]{xcolor}
\usepackage{minitoc}
\usepackage{array}

\definecolor{yellow}{cmyk}{0,0,1,0}
\renewcommand{\arraystretch}{1.4}
\newcommand{\R}{\mathbb{R}}

\usepackage{colortbl}

% Page size
\setlength{\oddsidemargin}{-0.5in}
\setlength{\evensidemargin}{-0.5in}
\setlength{\textheight}{10.25in}
\setlength{\textwidth}{7.0in}
\setlength{\topmargin}{-1.35in}

\renewcommand{\arraycolsep}{3pt}


\input color_flatex

\begin{document}
\pagestyle{empty}

\resetsteps      % Reset all the commands to create a blank worksheet  

% Define the operation to be computed

\renewcommand{\operation}{ \left[ C \right] := \mbox{\sc syrk\_ac\_blk\_var3}( A, C ) }

\renewcommand{\routinename}{\operation}

% Step 1a: Precondition 

\renewcommand{\precondition}{
  C = \widehat{C}
}

% Step 1b: Postcondition 

\renewcommand{\postcondition}{ 
  \left[C \right]
  =
  \mbox{syrk\_ac}( A, \widehat{C} )
}

% Step 2: Invariant 
% Note: Right-hand side of equalities must be updated appropriately

\renewcommand{\invariant}{
  \FlaTwoByTwo{C_{TL}}{C_{TR}}
               {C_{BL}}{C_{BR}} =
  \FlaTwoByTwo{\widehat{C}_{TL}}{\widehat{C}_{TR}}
               {\widehat{C}_{BL}}{A_R^TA_R + \widehat{C}_{BR}}
}

% Step 3: Loop-guard 

\renewcommand{\guard}{
  n( A_R ) < n( A )
}

% Step 4: Initialize 

\renewcommand{\partitionings}{
  $
  A \rightarrow
  \FlaOneByTwo{A_L}{A_R}
  $
,
  $
  C \rightarrow
  \FlaTwoByTwo{C_{TL}}{C_{TR}}
              {C_{BL}}{C_{BR}}
  $
}

\renewcommand{\partitionsizes}{
$ A_R $ has $ 0 $ columns,
$ C_{BR} $ is $ 0 \times 0 $
}

% Step 5a: Repartition the operands 

\renewcommand{\blocksize}{b}

\renewcommand{\repartitionings}{
$  \FlaOneByTwo{A_L}{A_R}
\rightarrow  \FlaOneByThreeL{A_0}{A_1}{A_2}
$
,
$  \FlaTwoByTwo{C_{TL}}{C_{TR}}
              {C_{BL}}{C_{BR}}
  \rightarrow
  \FlaThreeByThreeTL{C_{00}}{C_{01}}{C_{02}}
                    {C_{10}}{C_{11}}{C_{12}}
                    {C_{20}}{C_{21}}{C_{22}}
$}

\renewcommand{\repartitionsizes}{
$ A_1 $ has $ b $ columns,
  $ C_{11} $ is $ b \times b $}

% Step 5b: Move the double lines 

\renewcommand{\moveboundaries}{
$  \FlaOneByTwo{A_L}{A_R}
\leftarrow  \FlaOneByThreeR{A_0}{A_1}{A_2}
$
,
$  \FlaTwoByTwo{C_{TL}}{C_{TR}}
              {C_{BL}}{C_{BR}}
  \leftarrow
  \FlaThreeByThreeBR{C_{00}}{C_{01}}{C_{02}}
                    {C_{10}}{C_{11}}{C_{12}}
                    {C_{20}}{C_{21}}{C_{22}}
$}

% Step 6: State after repartitioning
% Note: The below needs editing!!!

\renewcommand{\beforeupdate}{
  $\FlaThreeByThreeTL{C_{00}}{C_{01}}{C_{02}}
                    {C_{10}}{C_{11}}{C_{12}}
                    {C_{20}}{C_{21}}{C_{22}}
  $
  =
  $\FlaThreeByThreeTL{C_{00}}{C_{01}}{C_{02}}
                    {C_{10}}{C_{11}}{C_{12}}
                    {C_{20}}{C_{21}}{A_2^TA_2 + \widehat {C}_{22}}
  $
}

% Step 7: State after moving of double lines
% Note: The below needs editing!!!

\renewcommand{\afterupdate}{
  $\FlaThreeByThreeBR{C_{00}}{C_{01}}{C_{02}}
                    {C_{10}}{C_{11}}{C_{12}}
                    {C_{20}}{C_{21}}{C_{22}}
  $
  =
  $\FlaThreeByThreeBR{C_{00}}{C_{01}}{C_{02}}
                    {C_{10}}{A_1^TA_1 + \widehat {C}_{11}}{A_1^TA2 + \widehat{C}_{12}}
                    {C_{20}}{C_{21}}{A_2^TA_2 + \widehat {C}_{22}}
  $
}

% Step 8: Insert the updates required to change the 
%         state from that given in Step 6 to that given in Step 7
% Note: The below needs editing!!!

\renewcommand{\update}{
$
  \begin{array}{l}
    C_{12} = A_1^TA2 + C_{12} \\
    C_{11} = A_1^TA_1 + C_{11} \\
  \end{array}
$
}












\begin{center}
	\FlaWorksheet
\end{center}

\newpage

\begin{figure}[p]
\begin{center}
	\FlaWorksheetZero
\end{center}
\end{figure}

\begin{figure}[p]
\begin{center}
	\FlaWorksheetOne
\end{center}
\end{figure}

\begin{figure}[p]
\begin{center}
	\FlaWorksheetTwo
\end{center}
\end{figure}

\begin{figure}[p]
\begin{center}
	\FlaWorksheetThree
\end{center}
\end{figure}

\begin{figure}[p]
	\begin{center}
	\FlaWorksheetFour
\end{center}
\end{figure}

\begin{figure}[p]
	\begin{center}
	\FlaWorksheetFive
\end{center}
\end{figure}

\begin{figure}[p]
	\begin{center}
	\FlaWorksheetSix
\end{center}
\end{figure}

\begin{figure}[p]
	\begin{center}
	\FlaWorksheetSeven
\end{center}
\end{figure}

\begin{figure}[p]
	\begin{center}
	\FlaWorksheetEight
\end{center}
\end{figure}

\begin{figure}[p]
	\begin{center}
	\FlaWorksheetNine
\end{center}
\end{figure}

\begin{figure}[p]
\begin{center}
	\FlaAlgorithm
\end{center}
\end{figure}

\end{document}