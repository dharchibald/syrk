\chapter{Example: $ B := L B $ --  \large Team: Robert van de Geijn}



\section{Operation}

Consider the operation
\[
B := L B 
\]
where $ L $ is a $ m \times m $ lower triangular matrix and $ B $ is a $ m \times n $ matrix.
This is a special case of triangular 
matrix-matrix multiplication, 
with the {\sc l}ower triangular matrix on the {\sc l}eft, 
and the triangular matrix is {\sc n}ot transposed.
We will refer to this operation
as {\sc Trmm\_llnn} where the {\sc llnn} stands for
\underline{l}eft
\underline{l}ower
\underline{n}o-transpose
\underline{n}onunit diagonal.
The {\sc n}onunit diagonal means we will use the entries of the matrix that are stored on the diagonal.  In some other cases, the entries on the diagonal may implicitly be taked to all equal one ({\sc u}nit).

\section{Precondition and postcondition}

In the precondition 
\[
B = \widehat B
\]
$ \widehat B $ denotes the original contents of $ B $.
This allows us to express the state upon completion, the postcondition, as
\[
B = L \widehat B.
\]
It is implicitly assumed that $ L $ is nonunit lower triangular.
\section{Partitioned Matrix Expressions and loop invariants}

There are two PMEs for this operation.

\subsection{PME 1}

To derive the second PME, partition
\[
L \rightarrow
\left(\begin{array}{c I c}
L_{TL} & 0 \\ \whline
L_{BL} & L_{BR}
\end{array}\right)
,
	\quad \mbox{and} \quad
B \rightarrow 
\left(\begin{array}{c}
B_T \\ \whline
B_B
\end{array}\right).
\]
Substituting these into the postcondition
yields
\[
\left(\begin{array}{c}
B_T \\ \whline
B_B
\end{array}\right)
=
\left(\begin{array}{c I c}
L_{TL} & 0 \\ \whline
L_{BL} & L_{BR}
\end{array}\right)
\left(\begin{array}{c}
\widehat B_T \\ \whline
\widehat B_B
\end{array}\right)
\]
or, equivalently,
\[
\left(\begin{array}{c}
B_T \\ \whline
B_B
\end{array}\right)
=
\left(\begin{array}{c}
L_{TL} \widehat B_T \\ \whline
L_{BL} \widehat B_T + L_{BR} \widehat B_B
\end{array}\right),
\]
which we refer to as the first PME for this operations.

From this, we can choose two loop invariants:
\begin{description}
	\item
	{\bf Invariant 1:}
	$
	\left(\begin{array}{c}
	B_T \\ \whline
	B_B
	\end{array}\right)
	 = 
	 \left(\begin{array}{c}
	 \widehat B_T \\ \whline
	 L_{BR} \widehat B_B
	 \end{array}\right)
	$. \\
	(The top part has been left alone and the bottom part has been partially computed).
	\item
	{\bf Invariant 2:}
	$
	\left(\begin{array}{c}
	B_T \\ \whline
	B_B
	\end{array}\right) = 
	\left(\begin{array}{c}
	\widehat B_T \\ \whline
	L_{BL} \widehat B_T + L_{BR} \widehat B_B
	\end{array}\right)
	$. \\
	(The top part has been left alone and the bottom part has been completely computed).
\end{description}

\subsection{PME 2}

To derive the second PME, partition
\[
B \rightarrow \left(\begin{array}{c I c}
B_L & B_R 
\end{array}\right)
\]
and do not partition $ L $.
Substituting these into the postcondition
yields
\[
\left(\begin{array}{c I c}
B_L & B_R 
\end{array}\right)
=
L 
\left(\begin{array}{c I c}
\widehat B_L & \widehat B_R 
\end{array}\right)
\]
or, equivalently,
\[
\left(\begin{array}{c I c}
B_L & B_R 
\end{array}\right)
=
\left(\begin{array}{c I c}
L \widehat B_L & L \widehat B_R 
\end{array}\right),
\]
which we refer to as the second PME.

From this, we can choose two more loop invariants:
\begin{description}
	\item
	{\bf Invariant 3:}
	$
	\left(\begin{array}{c I c}
	B_L & B_R 
	\end{array}\right) =
	\left(\begin{array}{c I c}
	L \widehat B_L & \widehat B_R 
	\end{array}\right)$.\\
	 (The left part has been completely finished and the right part has been left untouched).
	\item
	{\bf Invariant 4:}
	$
	\left(\begin{array}{c I c}
	B_L & B_R 
	\end{array}\right) =
	\left(\begin{array}{c I c}
	\widehat B_L & L \widehat B_R 
	\end{array}\right)$. \\
	(The left part has been completely finished and the right part has been left untouched).
\end{description}

\subsection{Notes}

How do I decide to partition the matrices in the postcondition?

\begin{itemize}
	\item
	Pick a matrix (operand), any matrix.  
	\item 
	If that matrix has 
	\begin{itemize}
		\item 
	a triangular structure (in storage), then you want to either partition is into four quadrants, or not at all.  Symmetric matrices and triangular matrices have a triangular structure (in storage).
		\item
	no particular structure, then you partition it vertically (left-right), horizontally (top-bottom), or not at all.
	\end{itemize}
	\item
	Next, partition the other matrices similarly, but conformally (meaning the 
	resulting multiplications with the parts are legal).
\end{itemize}
Take our problem here:  $ B := L B $.
Start by partitioning $ L $ in to quadrants:
\[
B = 
\left(\begin{array}{c I c}
L_{TL} & 0 \\ \whline
L_{BL} & L_{BR}
\end{array}\right)
		\widehat B.
\]
Now, the way partitioned matrix multiplication works, this doesn't make sense:
\[
B = 
\begin{array}[t]{c}
\underbrace{
\left(\begin{array}{c I c}
L_{TL} & 0 \\ \whline
L_{BL} & L_{BR}
\end{array}\right)
		\widehat B.
	}\\
	\left(\begin{array}{c}
	L_{TL} \times \mbox{something} + 0 \times \mbox{something} \\ \whline
	L_{BL} \times \mbox{something} + L_{BR} \times \mbox{something}
	\end{array}\right)
	\end{array}.
\]
So, we need to also partition $ B $ into a top part and a bottom part:
\[
\left(\begin{array}{c}
B_T \\ \whline
B_B
\end{array}\right)
= 
\begin{array}[t]{c}
\underbrace{
	\left(\begin{array}{c I c}
	L_{TL} & 0 \\ \whline
	L_{BL} & L_{BR}
	\end{array}\right)
		\left(\begin{array}{c}
		\widehat B_T \\ \whline
		\widehat B_B
		\end{array}\right).
	}\\
	\left(\begin{array}{c}
	L_{TL} \widehat B_T + 0 \times \widehat B_B\\ \whline
	L_{BL}  \widehat B_T + L_{BR}  \widehat B_B
	\end{array}\right)
	\end{array}
\]

Alternatively, what if you don't partition $ L $?  You have to partition {\em something} so let's try partitioning $ B $:
\[
\left(\begin{array}{c}
B_T \\ \whline
B_B
\end{array}\right)
=
L 
\left(\begin{array}{c}
\widehat B_T \\ \whline
\widehat B_B
\end{array}\right).
\]
But that doesn't work...
Instead
\[
\left(\begin{array}{c I c}
B_L & B_R 
\end{array}\right)
=
L 
\left(\begin{array}{c I c}
\widehat B_L & \widehat B_R 
\end{array}\right)
=
\left(\begin{array}{c I c}
L \widehat B_L & L \widehat B_R 
\end{array}\right)
\]
works just fine.  

\section{Deriving all unblocked algorithms}

The below table summarizes all loop invariants, with links to all files related to this operation.

The worksheet and code skeletons were genered using 
 the \href{http://edx-org-utaustinx.s3.amazonaws.com/UT1401x/LAFFPfC/Spark/index.html}{\ding{42} Spark webpage}.
 

\begin{center}
	\begin{tabular}{| c | l I c | l |} \hline
		& Invariant & Derivations &Implementations \\ \whline
		1 & 
		$
		\left(\begin{array}{c}
		B_T \\ \whline
		B_B
		\end{array}\right) = 
		\left(\begin{array}{c}
		\widehat B_T \\ \whline
		L_{BR} \widehat B_B
		\end{array}\right)
		$
		&
		\href{trmm_llnn/Derivations/trmm_llnn_unb_var1.pdf}
		{PDF}
		&
		\begin{minipage}{0.3\textwidth}
		\href{trmm_llnn/flameatlab/trmm_llnn_unb_var1.mlx}
		{trmm\_llnn\_unb\_var1.mlx}\\
		\href{trmm_llnn/FLAMEC/trmm_llnn_unb_var1.c}
		{trmm\_llnn\_unb\_var1.c}
	    \end{minipage}
	    \\ \hline
	    2 & 
	    $
	    \left(\begin{array}{c}
	    B_T \\ \whline
	    B_B
	    \end{array}\right) = 
	    \left(\begin{array}{c}
	    \widehat B_T \\ \whline
	    L_{BL} B_T + L_{BR} \widehat B_B
	    \end{array}\right)
	    $
	    &
	    \href{trmm_llnn/Derivations/trmm_llnn_unb_var2.pdf}
	    {PDF}
	    &
	    \begin{minipage}{0.3\textwidth}
	    	\href{trmm_llnn/flameatlab/trmm_llnn_unb_var2.mlx}
	    	{trmm\_llnn\_unb\_var2.mlx}\\
	    	\href{trmm_llnn/FLAMEC/trmm_llnn_unb_var2.c}
	    	{trmm\_llnn\_unb\_var2.c}
	    \end{minipage}
	    \\ \hline
	    3 & 
	    $
	    \left( \begin{array}{c I c}
	    B_L & B_R
	    \end{array}\right) = 
	    \left( \begin{array}{c I c}
	    L \widehat B_L & \widehat B_R
	    \end{array}\right)
	    $
	    &
	    \href{trmm_llnn/Derivations/trmm_llnn_unb_var3.pdf}
	    {PDF}
	    &
	    \begin{minipage}{0.3\textwidth}
	    	\href{trmm_llnn/flameatlab/trmm_llnn_unb_var3.mlx}
	    	{trmm\_llnn\_unb\_var3.mlx}\\
	    	\href{trmm_llnn/FLAMEC/trmm_llnn_unb_var3.c}
	    	{trmm\_llnn\_unb\_var3.c}
	    \end{minipage}
	    \\ \hline
	    4 & 
	    $
	    \left( \begin{array}{c I c}
	    B_L & B_R
	    \end{array}\right) = 
	    \left( \begin{array}{c I c}
	    \widehat B_L & L \widehat B_R
	    \end{array}\right)
	    $
	    &
	    \href{trmm_llnn/Derivations/trmm_llnn_unb_var4.pdf}
	    {PDF}
	    &
	    \begin{minipage}{0.3\textwidth}
	    	\href{trmm_llnn/flameatlab/trmm_llnn_unb_var4.mlx}
	    	{trmm\_llnn\_unb\_var4.mlx}\\
	    	\href{trmm_llnn/FLAMEC/trmm_llnn_unb_var4.c}
	    	{trmm\_llnn\_unb\_var4.c}
	    \end{minipage}
	    \\ \hline
	\end{tabular}
\end{center}

\end{document}